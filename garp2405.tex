\documentclass[12pt]{article}
\renewcommand{\baselinestretch}{2} %Double space
\usepackage[utf8]{inputenc}
\usepackage{pslatex}
\usepackage{amsmath}
\usepackage{graphicx}
\usepackage{geometry}
\geometry{legalpaper, lmargin=0.7in, bmargin=1in}
\usepackage[english, french]{babel}
\selectlanguage{french}

\setlength\parindent{0pt}% globally suppress indentation

\begin{document}
%%%%%%%%%%%%%%
%page  titre en caractères plus large
%%%%%%%%%%%%%%
\begin{titlepage}   
	\large{
		\begin{center}
			UNIVERSITÉ DE SHERBROOKE\\Faculté de génie\\
			Département de génie électrique et génie informatique\\
			\vspace{3cm}
			{\LARGE\textbf{Ingénieur et Société}}\\
			\vspace{2cm}
			\LARGE{Procédural}\\
			\vspace{2cm}
			Présenté à\\l'équipe professorale de la session S4\\
			\vspace{2cm}
			Produit par\\Philippe Garneau\\
			\vspace{1cm}
			\vfill{17 juin 2017 - Sherbrooke}
		\end{center}
	}
\end{titlepage}
\newpage
%%%%%%%%%%%%%%
%Table des matières
%%%%%%%%%%%%%%
\tableofcontents

\newpage
\section{Introduction}
La surveillance électronique est un phénomène récent qui sème la controverse, peu importe le secteur qu’il affecte. Dans ce procédural, le sujet de la surveillance électronique sera appliqué au principe de la neutralité du Net qui est une discussion toujours en cours aux États-Unis. En premier lieu, une explication de ce qu’est la neutralité du net et de son historique sera donnée. En deuxième lieu, il y aura une présentation des différents acteurs en jeux et la controverse entourant la neutralité du Net. En troisième lieu, il y aura une présentation de diverses théories discutées en classe et de leur application à la controverse de la neutralité du net. Finalement, un diagnostic sera appliqué à l’implémentation de la neutralité du net.

\section{Développement}
\subsection{Description du cas}
La neutralité du Net est un principe qui dicte que les fournisseurs d’Internet ne devraient pas avoir le droit de bloquer ou ralentir leur service dépendamment du contenu consommé et ne devraient pas laisser d’autres compagnies payer pour que leur contenu se rende plus rapidement aux clients. Depuis 1980, la "Federal Communications Commission" ou FCC a divisé les services de communication en deux catégories. Les lignes téléphoniques avec leur transmission "pure" entrent dans la catégorie de base tandis que les services comme l’Internet entrent dans la catégorie avancée. Seuls les services de base sont sujets aux "common carrier laws" qui empêchent les fournisseurs de discriminer contre ou refuser le service à des clients. En 1996, le "New Telecommunications Act" amène des termes plus spécifiques afin de catégoriser les différents services offerts aux consommateurs. Les services de base sont maintenant appelés "Title II telecommunications carriers" et peuvent être définis par le fait qu’ils transmettent de l’information seulement. Les services dans la catégorie avancée sont maintenant classés en tant que "Title I information service providers". Par exemple, les compagnies offrant des services Digital Subscription Line (DSL) font maintenant partie de la catégorie "Title II" tandis que des portails de style AOL restent dans la catégorie "Title I". En 2002, après une confusion légale due à l’ambiguïté des mots utilisés pour décrire ce qu’est un "information service provider", la FCC catégorise les compagnies comme Comcast et Verizon en tant que fournisseuses d’information, donc ceux-ci ne sont plus dans l’obligation de respecter les "common carrier laws" qui protègent les consommateurs. En 2005, la FCC essaye de redéfinir ce qu’est un "telecommunication carrier" pour inclure les nombreuses compagnies offrant des services Internet, mais leur décision sera rejetée en cours. C’est à ce moment que la FCC décide de créer une série de réglementations qu’elle appelle "open internet rules". La première règle dicte qu’aucun fournisseur ne peut bloquer l’accès à du contenu légal, des applications et des services. La deuxième règle dicte qu’aucun fournisseur ne peut intentionnellement ralentir le débit de l’Internet afin qu’un utilisateur le reçoive plus lentement. La troisième règle dicte que les fournisseurs ne peuvent prioriser aucune forme de contenu. En 2007, Comcast est reconnu coupable de ralentir le débit d’Internet de ses utilisateurs lorsqu’ils utilisent un site nommé BitTorrent. Cette révélation amènera la FCC à ordonner à Comcast de changer ses politiques, mais Comcast amènera sa cause en court disant que la FCC n’a aucune autorité pour censurer Comcast. En 2011, les "open internet rules" définies par la FCC sont mises en place, mais ça ne sera pas long que d’autres fournisseurs d’Internet comme Verizon et MetroPCS amènent leur cause en court et en 2014, la court fédérale donne raison aux fournisseurs d’Internet. Comme mentionné dans ce bref historique de la situation, les principaux groupes et acteurs impliqués sont la FCC, les fournisseurs d’Internet et les clients. D’un côté, il y a les fournisseurs comme Comcast qui surveillent en tout temps le contenu qui est consommé par ses clients et qui veulent augmenter leur marge de profits en limitant l’accès à certains types de contenu plus coûteux. Cela pourrait même aller plus loin où Comcast pourrait vendre l’information de ses clients à des compagnies marketing et celles-ci pourraient créer du contenu sur mesure pour ensuite demander à Comcast de prioriser ce contenu pour certains utilisateurs. D’un autre côté, il y a la FCC qui est une agence gouvernementale qui est censée avoir l’intérêt de la population à cœur. Par contre, étant une agence gouvernementale, la FCC peut être victime de l’influence que les grosses compagnies et leurs lobbyistes peuvent avoir sur l’élite politique. Finalement, il y a le pauvre consommateur qui doit subir les conséquences sans trop avoir un mot à dire.

\subsection{Description de la controverse}
La controverse dans le sujet de la neutralité du net peut provenir de différents aspects de la problématique. En premier lieu, il y a un problème d’espionnage que font les fournisseurs d’Internet dans la vie de leurs utilisateurs. Beaucoup de gens seraient très fâchés si on leur disait que tout ce qu’ils font est enregistré et analysé. On peut faire un lien avec la société autoritaire qu’on peut observer dans le livre "1984" par George Orwell. Une entité qui écoute et regarde en tout temps n’est pas nécessairement une bonne chose et peut apporter des conséquences très graves. Un argument souvent amené en faveur de l’espionnage du contenu Internet est le fait que si on ne fait rien d’illégal, pourquoi devrait-on être contre cet espionnage? C’est un bon argument lorsqu’on pense seulement au présent et ce que ces fournisseurs font avec ces informations. Par contre, si le principe d’espionnage électronique n’est plus considéré comme étant négatif, qu’est-ce qui empêcherait ces compagnies d’étendre leur influence sur plusieurs aspects de la vie quotidienne à l’aide de cette information. Par exemple, si je fais une recherche sur le web pour trouver la meilleure marque de couches pour bébés et que plus tard dans la journée, je vais au magasin et les couches "Huggies" sont en rabais seulement pour moi. Les couches "Huggies" ne sont pas nécessairement les meilleures, mais la compagnie "Huggies" aurait pu payer que savoir que le consommateur était à la recherche de couches et ainsi influencer la décision finale du consommateur en lui offrant un rabais. En deuxième lieu, il y a problème entre la relation de la FCC avec le public. En temps normal, un citoyen concerné pourrait divulguer son opinion à son représentant au congrès et ainsi potentiellement positivement influencer les décisions de la FCC, mais le problème est que les États-Unis n’ont pas de règles très strictes sur les dons que les compagnies ou lobbyistes peuvent faire à ces membres du congrès. Cela cause en quelque sorte un conflit d’intérêts, car la FCC est supposée défendre les citoyens, mais celle-ci est fortement influencée ou contrôlée par les membres du congrès qui sont fort problament corrompus par ces fournisseurs d’Internet. Finalement, il y a un problème entre la relation de la FCC et ces fournisseurs d’Internet. Comme décrit dans l’historique du cas à l’étude, les fournisseurs d’Internet s’opposent fortement aux règlementations que la FCC essaye de leur imposer depuis plusieurs années et n’ont pas peur de plaider leur cause en court. Ces compagnies ont beaucoup d’argent et peuvent plaider leur cause très longtemps en cour. Cela veut dire que le progrès au niveau du consommateur est très lent, car toute forme d’avancement est rencontrée avec un bâton dans les roues. Plus récemment, une forme de discours de coopération est provenue des fournisseurs d’Internet. Ceux-ci veulent abolir les règles courantes afin de les redéfinir pour le monde moderne. Cela fait peur, car cette renégociation pourrait donner l’avantage aux fournisseurs d’Internet grâce à l’aide du congrès et ainsi garantir la perte des consommateurs. Par contre, il ne faut pas voir toutes les compagnies comme étant contre la neutralité du net afin de maximiser leurs profits. Certaines compagnies comme "Netflix" et "Amazon" font partie de la défensive envers la neutralité du Net et on pourrait argumenter que sans celles-ci, le combat serait déjà perdu.

\subsection{Thématiques impliquées}
\subsubsection{Chronos}
Premièrement, le concept de chronos est impliqué dans cette controverse. Comme mentionné dans les notes de cours, chronos représente le temps qui se mesure et qui est précisément compté. Dans notre controverse, nous avons une compagnie Comcast qui désire avoir le contrôle sur le contenu de ses utilisateurs. En surveillant le contenu consommé par ses utilisateurs, Comcast peut déterminer exactement quels clients sont rentables et quels clients coûtent trop cher à supporter. Par exemple, si un client paye un montant élevé pour avoir accès à une quantité illimitée de contenu chaque mois, mais n’utilise ce service que pour visiter des pages web et potentiellement téléverser quelques fichiers par mois, ce client est très rentable. Celui-ci est très rentable, car il minimise le temps que l’infrastructure de Comcast subit un haut niveau de stress. Un autre exemple serait un client qui paye pour le même forfait, mais celui-ci est un grand adepte du cinéma. Ce client peut téléverser des vidéos toute la journée, et ce pendant tous les jours du mois. Cela cause un très grand stress sur l’infrastructure de Comcast, mais la compagnie reçoit le même montant d’argent que le client précédent. C’est pour cela que Comcast a un intérêt de surveiller les habitudes de consommation de ses clients, car cette consommation se traduit en temps de haute charge sur l’infrastructure et cela se traduit en coûts plus élevés. Une autre application de chronos pourrait se produire si les fournisseurs d’internet offraient des vitesses plus élevées sur du contenu spécifique. Une tierce compagnie pourrait payer un fournisseur d’internet pour afficher son contenu avec des vitesses préférentielles. C’est en surveillant ce contenu qu’une compagnie comme Comcast pourrait mesurer exactement le temps que ses clients ont regardé ce contenu promotionnel et ainsi charger un prix variant à ces tierces compagnies.

\subsubsection{Diatribè}
Le concept de diatribè peut être utilisé pour illustrer ce que vivent les consommateurs dans cette controverse. En tant que consommateurs, nous avons une dépendance envers l’accès à l’Internet et ces fournisseurs d’Internet sont notre seul point d’accès. Comme inscrit dans les notes de cours, diatribè réfère à ce qui s’impose à nous sans que nous soyons en position de riposter. Dans la controverse de la neutralité du net, les deux joueurs principaux sont les gros fournisseurs d’Internet comme Comcast et Verizon ainsi que la FCC. Normalement, on pourrait croire que les citoyens ont un mot à dire dans le débat considérant que la FCC est une organisation gouvernementale, mais ceci n’est pas le cas. Les fournisseurs d’Internet ont un très grand cercle d’influence pouvant aller jusqu’aux membres du Congrès américain et c’est cette influence qui prend le dessus sur l’opinion des citoyens. Connaissant ces conditions, le consommateur subit les conséquences de décisions prises par de riches hommes d’affaires qui n’ont que le profit en tête.

\subsubsection{La veille technologique}
La veille technologique est un type de transfert technologique indirect qui peut s’appliquer à ce cas, car c’est exactement ce que font les fournisseurs d’Internet. Ces fournisseurs analysent l’utilisation que font leurs clients afin de s’adapter et créer de nouveaux services. Cet espionnage électronique pourrait aussi être utilisé pour créer des transferts technologiques entre différentes compagnies qui n’œuvrent pas nécessairement dans le même domaine. Par exemple, si un nouveau site de réseau social est inventé et qu’un fournisseur Comcast détecte une quantité de trafic augmentant de plus en plus chaque jour, le fournisseur pourrait agir de plusieurs façons. Premièrement, il pourrait décider d’acheter la compagnie propriétaire du site de réseau social pour un prix raisonnable, mais avec l’intention de profiter de la popularité grandissante de ce site pour avoir un gros retour sur son investissement. Deuxièmement, le fournisseur pourrait offrir du service préférentiel aux utilisateurs de ce site en échange d’argent ou de technologie avec la compagnie propriétaire du site. Un autre bon exemple qui pourrait devenir réalité très bientôt si on observe l’achat de la chaîne d’épiceries "Wholefoods" par Amazon est l’arrivée de nouvelles façons d’acheter des produits à l’épicerie. Si mon fournisseur d’Internet est capable de voir toutes les choses qui m’intéressent et que je recherche sur l’Internet, une forme de marketing personnalisé pourrait se produire. Par exemple, si je fais une recherche sur une nouvelle recette de porc à la moutarde, cette information pourrait être enregistrée pour m’offrir des spéciaux à l’épicerie sur le porc et la moutarde. Certaines entreprises comme Choix du Président offrent déjà un service similaire où ils analysent les habitudes d’achat de leurs consommateurs afin de leur offrir des offres sur mesure, mais imaginons si ces offres pouvaient provenir d’autres sources de la vie quotidienne du consommateur. Ce besoin de trouver une nouvelle façon de magasiner amène une progression technologique.

\subsection{Diagnostic}
Selon moi, la neutralité du net devrait absolument être mise en place, mais certaines modifications devraient y être apportées. Si on observe les règles définies dans les "Open Internet Rules", je suis entièrement d’accord avec leur implémentation. Je ne crois pas qu’un consommateur qui paye pour un service devrait avoir ce service modifié ou enlevé dépendamment de l’opinion des dirigeants de la compagnie offrant le service. Toute forme de trafic sur l’Internet devrait être considérée comme égal et ne devrait pas faire face au jugement d’une tierce personne. Il est vrai que certaines formes de trafic causent un stress élevé sur l’infrastructure des fournisseurs d’Internet, mais ceux-ci devraient se protéger à l’aide de règlementations sur la quantité du contenu et non le type de contenu. Les gens voulant consommer de plus grosses sommes de contenu pourront le faire tant et aussi longtemps qu’ils payent pour ce service. Je suis aussi entièrement contre la priorisation de contenu par les fournisseurs d’Internet. Ce n’est pas parce qu’une tierce compagnie a payé que le consommateur devrait recevoir un accès privilégié à ce contenu. Ayant des créateurs de contenu au même niveau encourage le progrès, car c’est ce qui va différencier le contenu. Même si je suis pour la neutralité du Net, je ne suis pas nécessairement contre la forme d’espionnage électronique que ces fournisseurs d’Internet font. Je trouve qu’il y a plusieurs applications de cette intrusion qui pourraient être bénéfiques aux consommateurs. Par exemple, l’expérience que j’ai décrit dans la section de veille technologique où un consommateur pourrait avoir des rabais ou simplement avoir un accès plus rapide aux produits qu’il recherche vraiment pourrait être très positif. Je suis d’avis avec l’opinion que tant que je ne fais rien d’illégal, je n’ai rien à craindre de cet espionnage, donc c’est un plus pour moi.

\section{Conclusion}
En conclusion, la neutralité du net est une nécessité afin de protéger les droits du consommateur, mais je présente une certaine ouverture envers les avantages qu’une forme de surveillance informatique pourrait apporter aux utilisateurs du web. Ce n’est pas un sujet simple et il faut faire attention de bien observer les deux côtés de la médaille avant de prendre une décision.

\begin{thebibliography}{12}
\bibitem{fcc_site}
FCC.
\textit{The Open Internet}
\\\texttt{https://www.fcc.gov/consumers/guides/open-internet}

\bibitem{verge}
Nilay Patel.
\textit{The Wrong Words: How the FCC Lost Net Neutrality and Could Kill The Internet}
\\\texttt{https://www.theverge.com/2014/1/15/5311948/net-neutrality-and-the-death-of-the-internet}

\bibitem{verge_2}
Nilay Patel.
\textit{Everything Verizon says in this terrible video about net neutrality vs. the truth}
\\\texttt{https://www.theverge.com/2017/5/2/15520818/verizon-net-neutrality-craig-silliman-truth}

\bibitem{arstechnica}
Jon Brodkin.
\textit{Comcast and other ISPs celebrate imminent death of net neutrality rules}
\\\texttt{https://arstechnica.com/tech-policy/2017/04/isps-claim-to-love-net-neutrality-while-praising-death-of-net-neutrality-rules/}

\bibitem{wiki}
Wikipedia.
\textit{Net neutrality in the United States}
\\\texttt{https://en.wikipedia.org/wiki/Net\textunderscore neutrality\textunderscore in\textunderscore the\textunderscore United\textunderscore States}

\bibitem{jdm}
Pierre Trudel.
\textit{La neutralité d'internet}
\\\texttt{http://www.journaldemontreal.com/2016/06/02/la-neutralite-dinternet}

\bibitem{jr}
John Wihbey.
\textit{The net neutrality debtate and underlying dynamics: Research perspectives}
\\\texttt{https://journalistsresource.org/studies/society/internet/net-neutrality-debate-underlying-dynamics-research-perspectives}

\bibitem{research}
Yiannis Yiakoumis, Sachin Katti, Nick McKeown.
\textit{Neutral Net Neutrality}
\\\texttt{https://yuba.stanford.edu/~yiannis/docs/sigcomm-neutrality.pdf}


\end{thebibliography}

\end{document}

